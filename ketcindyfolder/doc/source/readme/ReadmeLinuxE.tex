\documentclass{article}
\usepackage{ketpic,ketlayer}
\usepackage{amsmath,amssymb}
\usepackage{graphicx}
\usepackage{xcolor}
\usepackage{bm,enumerate}
\usepackage[colorlinks=true,urlcolor=blue]{hyperref}

\setmargin{20}{20}{20}{20}

\renewcommand{\labelitemi}{$\cdot$}
\renewcommand{\labelitemii}{$\cdot$}

\begin{document}

\begin{center}
How to install KeTCindy(Linux)
\end{center}

\hfill modified\ :\ \today

\begin{enumerate}[\bf\large 1.]

\item Install Cinderella, R, Maxima and Evince.
 \item \url{https://beta.cinderella.de}  (Cinderella)
    \begin{itemize}
    \item[Rem)]Download `Unix Install' and execute on terminal.\\
    \hspace*{10mm}\verb|sudo bash Cinderella_unix_2_8.sh|
    \item[Rem)]if Java has not been installed, install it.\\
     \hspace*{10mm}\verb|sudo apt install openjdk-7-jre|(or 8)
    \end{itemize}
   \item R and Maxima can be installed:\\
\hspace*{10mm}\verb|sudo apt install r-base maxima|
    \begin{itemize}
    \item[Rem)]HPs of R and Maxima are:\\
    \hspace*{10mm}\url{https://cran.r-project.org}\\
    \hspace*{10mm}\url{https://sourceforge.net/projects/maxima/files}
    \end{itemize}
\item Install TeX if any TeX has not been installed.
  \begin{enumerate}[(1)]
  \item TeXLive is recommended.
    \begin{itemize}
    \item Files necessary for KeTCindy are already implemented (2018 or later).
    \end{itemize}
  \item KeTTeX is a light-weight version of TeXLive and downloadable from\\
  \hspace*{6mm}\url{https://www.dropbox.com/s/vg8p07832e9hzlk/KeTTeX-linux-20171022.tar.xz?dl=0}
 \end{enumerate}

\item Install KeTCdindy
  \begin{enumerate}[(1)]
  \item Download ketcindy from CTAN(\url{https://ctan.org})\\
  \hspace*{10mm}Search ketcindy \verb|>| Pack­age ketcindy \verb|>| download
    \begin{itemize}
    \item[Rem)]The latest version can be download from Repository:\\
        \hspace*{5mm}\url{https://github.com/ket­pic/ketcindy}\\
        \hspace*{10mm}Clone or Download \verb|>| Download ZIP
    \end{itemize}
  \item Open ketcindy(-master)/forLinux.
  \item Open setketcindy.sh with a text editor and confirm/modify the paths.
    \begin{itemize}
    \item Execute with terminal:\\
    \hspace*{10mm}\verb|sudo bash setketcindy.sh|
    \item Contents of scripts will be copied into TeX.
    \item ketcindystyle files will be copied and mktexlsr will be executed.
    \item In Cinderella/PlugIns\\
    \hspace*{5mm}KetcindyPluign.jar will be copied.\\
    \hspace*{5mm}ketcindy.ini will be generated .
    \end{itemize}
  \item Open setwork.sh with a text editor and confirm/modify it.
    \begin{itemize}
    \item Execute with terminal:\\
    \hspace*{10mm}\verb|bash setwork.sh|
    \item TeX(typeset) will be usually latex,xelatex or pdflatex.
    \item Contents of “work” will be copied into "ketcindy"
    \item \verb|.ketcindy.conf| will be also generated in User's home.\\
    \hspace*{10mm}You can change the setting of PasthT, Mackc, etc.
    \item Template of "ketcindy.conf" will be also copied to work directory.
    \item Configuration files are read in order of 
      \begin{enumerate}[1)]
      \item ketoutset.txt
      \item ketcindy.conf in User's home
      \item ketcindy.conf in the work folder.
      \end{enumerate}
    \end{itemize}
  \end{enumerate}

\item Test run of KeTCindy
\begin{enumerate}[(1)]
  \item Test template1basic.cdy in ketcindy folder(work directory).
    \begin{itemize}
    \item Execute on terminal:
      \hspace*{10mm}\verb|cd (to the path of ketcindy)|\\
      \hspace*{10mm}\verb|Cinderella2 template1basic.cdy|
    \item a frame in white will appear on the screen.
    \end{itemize}
  \item Press "Figure" button at the top left, then the final PDF output is displayed. 
 \end{enumerate}

\item Set Texworks if necessary. 
  \begin{itemize}
  \item Downloadable from \url{https://github.com/TeXworks/texworks/releases/}.
  \end{itemize}

\item Install \verb|gcc| for drawings of surface.

\end{enumerate}

\end{document}