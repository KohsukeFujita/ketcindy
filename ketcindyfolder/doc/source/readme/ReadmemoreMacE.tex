\documentclass{article}
\usepackage{ketpic,ketlayer}
\usepackage{amsmath,amssymb}
\usepackage{graphicx}
\usepackage{xcolor}
\usepackage{bm,enumerate}
\usepackage[colorlinks=true,urlcolor=blue]{hyperref}

\setmargin{20}{20}{20}{20}

\renewcommand{\labelitemi}{$\cdot$}
\renewcommand{\labelitemii}{$\cdot$}

\begin{document}

\begin{center}
How to install KeTCindy(Mac)
\end{center}

\hfill modified\ :\ \today

\begin{enumerate}[\bf\large 1.]

\item Install Cinderella, R and Maxima.
 \begin{itemize}
 \item \url{https://beta.cinderella.de}  (Cinderella)
 \item \url{https://cran.r-project.org}   (R)
 \item \url{https://sourceforge.net/projects/maxima/files}  (Maxima)
 \end{itemize}
\hspace*{5mm}Rem) Check they start correctly.

\item Install TeX if any TeX has not been installed.
  \begin{enumerate}[(1)]
  \item TeXLive is recommended.
    \begin{itemize}
    \item Files necessary for KeTCindy are already implemented (2018 or later).
    \end{itemize}
  \item kettex is a light-weight TeXLive.
    \begin{itemize}
    \item Download kettex.zip from\\
    \hspace*{10mm}\url{https://www.dropbox.com/s/dc4inuk06t07g26/kettex.dmg?dl=0}
    \item Double-click it and move kettex to \verb|/Applications|.
    \end{itemize}
  \item In case of other TeX, see {\bf 3.}(2).
 \end{enumerate}


\item Install KeTCdindy
  \begin{enumerate}[(1)]
  \item Download ketcindy from CTAN(\url{https://ctan.org})\\
  \hspace*{10mm}Search ketcindy \verb|>| Pack­age ketcindy \verb|>| download
    \begin{itemize}
    \item[Rem)]The latest version can be download from Repository:\\
        \hspace*{5mm}\url{https://github.com/ket­pic/ketcindy}\\
        \hspace*{10mm}Clone or Download \verb|>| Download ZIP
    \end{itemize}
  \item Open ketcindy(-master)/forMac.
    \begin{itemize}
    \item[Rem)]If the folder name includes spaces, remove them.
    \item[Rem)]If you use TeX other than TeXLIve/KeTTeX,
      \begin{itemize}
      \item Open setketcindy.command with a text editor.
      \item Edit paths in it.
      \end{itemize}
    \end{itemize}
  \item Double-click setketcindy.command.
    \begin{itemize}
    \item Password of an administrator is required.
    \item Control-click and select Terminal if necessary. 
    \item Contents of scripts will be copied into TeX.
    \item ketcindystyle files will be copied and mktexlsr will be executed.
    \item In Cinderella/PlugIns\\
    \hspace*{5mm}KetcindyPluign.jar will be copied.\\
    \hspace*{5mm}ketcindy.ini will be generated or remade(if selected).
    \item TeX(typeset) will be usually latex,xelatex or pdflatex.
    \end{itemize}
    \item Double-click setwork.command.
    \begin{itemize}
    \item[Rem)]Control-click and select Terminal if necessary. 
    \end{itemize}
    \begin{itemize}
    \item Work directory "ketcindy" will be generated in User's home.
    \item Contents of “work” will be copied into "ketcindy"
    \item \verb|.ketcindy.conf| will be also generated in User's home.\\
    \hspace*{10mm}You can change the setting of PasthT, Mackc, etc.
    \item Template of "ketcindy.conf" will be also copied to work directory.
    \item Configuration files are read in order of 
      \begin{enumerate}[1)]
      \item ketoutset.txt
      \item ketcindy.conf in User's home
      \item ketcindy.conf in the work folder.
      \end{enumerate}
    \end{itemize}
  \end{enumerate}

\item Test run of KeTCindy
 \begin{enumerate}[(1)]
  \item Double-click "template1basic.cdy" in work directory "ketcindy".\\
  \hspace*{10mm}Then a frame in white appear on the screen.  
  \item Press "Figure" button at the top left, then the final PDF output is displayed. 
    \begin{itemize}
    \item[Rem)]To close the window of Terminal when the process exits :
      \begin{itemize}
      \item Start \verb|Terminal|.\\
      \hspace*{10mm}Preferences \verb|>| Shell \verb|>| "Colse if the shell exited clearly"
      \end{itemize}
    \item[Rem)]It will be better to delete installer folder "ketcindy(-master)" .
    \end{itemize} 
  \end{enumerate}
\item Set Texworks if necessary. For example, in case of KeTTeX,
  \begin{itemize}
  \item Downloadable from \url{https://github.com/TeXworks/texworks/releases/}.
  \item Launch TeXworks,\\
  \hspace*{10mm}and choose Edit \verb|>| Preference \verb|>| Typeset.
  \item Push upper + and add\\
  \hspace*{5mm}\verb|/Applications/kettex/texlive/bin/x86_64-darwin|
  \item Select Xelatex or pdflatex.
  \end{itemize}

\item Install Xcode if you use gcc for drawings of surface.
  \begin{itemize}
  \item[Rem)]f you want to use only gcc, execute on Terminal:\\
    \hspace*{10mm}\verb|sudo xcode-select --install|
  \end{itemize}

\item Others
 \begin{itemize}
  \item Terminal command to allow to execute all aplications (Sierra or later)\\
\hspace*{20mm}sudo spctl --master-disable
 \end{itemize}
\end{enumerate}

\end{document}