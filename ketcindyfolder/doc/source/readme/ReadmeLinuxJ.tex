\documentclass{ujarticle}
\usepackage{ketpic,ketlayer}
\usepackage{amsmath,amssymb}
\usepackage{graphicx}
\usepackage{xcolor}
\usepackage{bm,enumerate}
\usepackage[dvipdfmx,colorlinks=true,urlcolor=blue]{hyperref}

\setmargin{20}{20}{20}{20}

\西暦

\renewcommand{\labelitemi}{・}
\renewcommand{\labelitemii}{・}

\begin{document}

\begin{center}
\ketcindy\ のインストール (Linux)
\end{center}

\hfill 修正日:\today

\begin{enumerate}[\bf\large 1.]
\item Cinderella, R, Maxima, Evince をインストールする.
 \begin{itemize}
 \item \url{https://beta.cinderella.de}  (Cinderella)
    \begin{itemize}
    \item[*]Unix Installをダウンロードして次を実行する.\\
   \verb|sudo bash Cinderella_unix_2_8.sh|
    \item[*]Javaがインストールされていなかったらインストールする.\\
   \verb|sudo apt install openjdk-7-jre|(または 8)
    \end{itemize}
    \item RとMaximaは次を実行すればインストールできる.\\
   \verb|sudo apt install r-base maxima|
    \begin{itemize}
    \item[*]R, MaximaのHPは\\
   \url{https://cran.r-project.org}   (R)\\
   \url{https://sourceforge.net/projects/maxima/files}  (Maxima)
    \end{itemize}
 \item Evince が入っていなければ,インストールする.\\
   \verb|sudo apt install evince|
 \end{itemize}

\item \TeX をインストールしていない場合はインストールする.
 \begin{enumerate}[(1)]
 \item TeXLiveを推奨
    \begin{itemize}
    \item 2018以降ではketcindyが既に入っている.
    \end{itemize}
\item KeTTeXはTeXLiveの軽量版で,以下からダウンロードできる.\\
    \hspace*{6mm}\url{https://www.dropbox.com/sh/79ofbls9nf0ywkj/AAA4KaH6MaFsaL0e-ACqw-0Ya?dl=0}

    \begin{itemize}
    \item どこか(例えばホーム)に置いて解凍する.
    \item ターミナルで以下を実行する.\\
   \verb|sudo bash ~/kettex/setkettex.sh|(ホームの場合)
    \end{itemize}

% \item 他のTeXの場合は,{\bf 3.}(2)を参照する.
 \end{enumerate}

\item KeTCindyのインストール
  \begin{enumerate}[(1)]
  \item ketcindyをCTAN(\url{https://ctan.org})からダウンロードする.\\
  \hspace*{10mm}ketcindyで検索 \verb|>| Package ketcindy \verb|>| download
    \begin{itemize}
    \item[*]最新版は,Repositoryのサイト\\
        \hspace*{10mm}\url{https://github.com/ket­pic/ketcindy}\\
       から以下のようにダウンロードできる.\\
        \hspace*{10mm}Clone or download \verb|>| Download ZIP
    \item[*]この場合は,ketcindy-masterになる.
    \end{itemize}
  \item ketcindy(-master)/forLinuxを開く.
  \item setketcindy.shをテキストエディタで開いて,パスを確認,修正する.
    \begin{itemize}
    \item ターミナルで以下を実行する.\\
   \verb|sudo bash setketcindy.sh|
    \item scriptsの中身が\TeX にコピーされる
    \item ketcindyのstyleファイルが\TeX にコピーされmktexlsrが実行される.
    \item CinderellaのPluginsにKetcindyPlugin.jarがコピーされketcindy.iniが作成される.
    \end{itemize}
  \item setwork.shをテキストエディタで開いて,パスを確認,修正する.
    \begin{itemize}
    \item ターミナルで以下を実行する.\\
   \verb|bash setketwork.sh|
    \item 作業ディレクトリketcindyがユーザホームに作成される.
    \item タイプセットの方法(\TeX の種類)\\
    \hspace*{10mm}通常は,platex (p)またはuplatex(u)を選ぶ.
    \item ketcindyフォルダにworkフォルダの中身がコピーされる.
    \item \verb|.ketcindy.conf|(不可視ファイルだが編集可能)がユーザホームに作成される.\\
    \hspace*{10mm}注)\TeX を切り替えるときなどはこのファイルを修正する.
    \item マニュアルもコピーされる.
    \item 作業ディレクトリにketincy.confの雛形がコピーされる.
    \item KeTCindyを立ち上げたとき,設定ファイルは次の順に読み込まれる.
      \begin{enumerate}[1)]
      \item ketoutset.txt
      \item ユーザホームの\verb|.ketcindy.conf|
      \item 作業ディレクトリketcindyの ketcindy.conf
      \end{enumerate}
    \end{itemize}
  \end{enumerate}

  \item KeTCindyのテストラン
    \begin{enumerate}[(1)]
    \item 作業ディレクトリketcindyにあるtemplate1basic.cdyを実行してみる.
      \begin{itemize}
      \item ターミナルで以下を実行\\
   \verb|cd (ketcindyのパス)|\\
   \verb|Cinderella2 template1basic.cdy|
      \item 画面に白い枠が出れば,ライブラリの読み込みは成功.
      \end{itemize}
    \item スクリーンの左上部にあるFigureボタンを押して,PDFが表示されれば成功.
  \end{enumerate}

\item TeXworksを設定する.
  \begin{itemize}
  \item \url{https://github.com/TeXworks/texworks/releases/} からダウンロードできる.
  \end{itemize}

\item gccのインストール
  \begin{itemize}
    \item 曲面描画のためには, gccが必要である.
  \end{itemize}
\end{enumerate}

\end{document}