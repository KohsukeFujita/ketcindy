\documentclass[papersize,a4paper,12pt]{article}
\usepackage{ketpic,ketlayer}
\usepackage{amsmath}
% \usepackage{amsmath,newtxmath}
%\usepackage[dvipdfmx]{graphicx,color}
\usepackage{graphicx,color}
\usepackage{wrapfig}
%\usepackage[dvipdfmx,bookmarks=false,colorlinks=true,linkcolor=blue]{hyperref}
\usepackage[bookmarks=false,colorlinks=true,linkcolor=blue]{hyperref}
\setmargin{20}{20}{15}{25}
\usepackage{setspace}
\usepackage{comment}
\usepackage{bm,enumerate}

\newcommand{\bs}{$\backslash$}

\newcommand{\br}[1]{\{#1\}}

\newenvironment{cmd}[2]{
\hypertarget{#2}{}
\begin{center}{\bf\large #1}\end{center}
\begin{description}
}{
\end{description}
\begin{flushright} \hyperlink{functionlist}{$\Rightarrow$Command List}\end{flushright}
}

% item command for this documentation
\newcommand{\itemket}[1]{
\item[\Ltab{27mm}{#1}]
}


\begin{document}
\title{Macros of ketpic.sty and ketlayer.sty}
\author{\ketcindy\ Project Team}
\maketitle

\begin{center}  - ver.1.1 -\end{center}

%\hypertarget{index}{}
%\tableofcontents

%\newpage

\section{Outlines}

\begin{itemize}
\item ketpic.sty, ketpic2e.sty(it is necessary in pict2e) are used for ketpic.
\item ketlayer.sty, ketlayer2e.sty(it is necessary in pict2e) are used for ketlayer.
\item \bs\verb|Width|, \bs\verb|Height|,\ \bs\verb|Depth| are defined.
\item Temporary counters \verb|ketictctra|,\ $\cdots$,\ \verb|ketpicctrj| are defined.
\item Package \verb|graphicx|, \verb|color| are required.

\end{itemize}


%========= Envilonment =====================
\section{Envilonment}

%------------layer--------------------------------
\begin{cmd}{layer}{layer}
\itemket{Usage}\verb|\begin{layer}[Ho]{W}{H}|\ $\cdots$\ \verb|\end{layer}|
\itemket{Description}This enviroment draws grids and adds a note or a figure.

\item[Details]\mbox{}\\
\Ltab{10mm}{W}Width of grids. The unit is \verb|mm|.

\Ltab{10mm}{H}Height of grids. The unit is \verb|mm|.\\
\hspace*{20mm}If \verb|H=0|, grids don't appear.\\
\hspace*{20mm}If \verb|H<0|, grids appear on the upside.


\itemket{Example} \mbox{}\\
\verb|\begin{layer}{120}{30}|\\
\verb|\putnotec{20}{10}{abc}|\\
\verb|\putnotes{60}{0}{%%% test.tex 2014-10-16 21:51
%%% test.sce 2014-10-16 21:50
{\unitlength=1cm%
\begin{picture}%
(   4.00000,   3.50000)(  -1.00000,  -1.00000)%
\special{pn 8}%
%
\settowidth{\Width}{A}\setlength{\Width}{0\Width}%
\settoheight{\Height}{A}\settodepth{\Depth}{A}\setlength{\Height}{\Depth}%
\put(1.0300,1.7900){\hspace*{\Width}\raisebox{\Height}{A}}%
%
%
\settowidth{\Width}{B}\setlength{\Width}{-1\Width}%
\settoheight{\Height}{B}\settodepth{\Depth}{B}\setlength{\Height}{-\Height}%
\put(-0.0500,-0.0500){\hspace*{\Width}\raisebox{\Height}{B}}%
%
%
\settowidth{\Width}{C}\setlength{\Width}{0\Width}%
\settoheight{\Height}{C}\settodepth{\Depth}{C}\setlength{\Height}{-\Height}%
\put(2.0500,-0.0500){\hspace*{\Width}\raisebox{\Height}{C}}%
%
%
\special{pa 0 0}\special{pa 386 -685}\special{pa 787 0}\special{pa 0 0}%
\special{fp}%
\end{picture}}%}|\\
\verb|\end{layer}|

\vspace{5mm}

\begin{layer}{120}{30}
\putnotec{20}{10}{abc}
\putnotes{60}{0}{%%% test.tex 2014-10-16 21:51
%%% test.sce 2014-10-16 21:50
{\unitlength=1cm%
\begin{picture}%
(   4.00000,   3.50000)(  -1.00000,  -1.00000)%
\special{pn 8}%
%
\settowidth{\Width}{A}\setlength{\Width}{0\Width}%
\settoheight{\Height}{A}\settodepth{\Depth}{A}\setlength{\Height}{\Depth}%
\put(1.0300,1.7900){\hspace*{\Width}\raisebox{\Height}{A}}%
%
%
\settowidth{\Width}{B}\setlength{\Width}{-1\Width}%
\settoheight{\Height}{B}\settodepth{\Depth}{B}\setlength{\Height}{-\Height}%
\put(-0.0500,-0.0500){\hspace*{\Width}\raisebox{\Height}{B}}%
%
%
\settowidth{\Width}{C}\setlength{\Width}{0\Width}%
\settoheight{\Height}{C}\settodepth{\Depth}{C}\setlength{\Height}{-\Height}%
\put(2.0500,-0.0500){\hspace*{\Width}\raisebox{\Height}{C}}%
%
%
\special{pa 0 0}\special{pa 386 -685}\special{pa 787 0}\special{pa 0 0}%
\special{fp}%
\end{picture}}%}\end{layer}
%% "FigE.tex" is copy of "addax2.tex"

\vspace{35mm}

\item[Remark]Set \verb|H=0| if placement of all components is proper.

\end{cmd}


%========= Macros =====================
\section{Macros}
%=-=-=-=-= Macros of ketpic =-=-=-=-=
\subsection{Macros of ketpic}
Macros of ketpic are used just like regular commands of \TeX.

%------------ ketpic ------------
\begin{cmd}{\bs ketpic}{ketpic}
\itemket{Usage}\verb|\ketpic|
\itemket{Description}This macro displays the logo of {\ketpic}.
\itemket{Examples} \verb|\ketpic|
\end{cmd}

%------------ ketcindy ------------
\begin{cmd}{\bs ketcindy}{ketcindy}
\itemket{Usage}\verb|\ketcindy|
\itemket{Description}This macro displays the logo of {\ketcindy}.
\itemket{Examples} \verb|\ketcindy|
\end{cmd}

%------------ tab ------------
\begin{cmd}{\bs Ltab, \bs Rtab, \bs Ctab}{tab}
\itemket{Usage}\verb|\Ltab{W}{S}|, \verb|\Rtab{W}{S}|, \verb|\Ctab{W}{S}|
\itemket{Description}This is tab macro.
\itemket{}\verb|\Ltab{W}{S}| secures the width of W and writes S by left justifying it.
\itemket{}\verb|\Rtab{W}{S}| secures the width of W and writes S by right justifying it.
\itemket{}\verb|\Ctab{W}{S}| secures the width of W and writes S at the center.
\end{cmd}

%------------ ketcalc ------------
\begin{cmd}{\bs ketcalcwidth, \bs ketcalcheight, \bs ketcalcdepth}{ketcalc}
\itemket{Usage}\verb|\ketcalcwidth[0]{C}|, \verb|\ketcalcheight[0]{C}|, \verb|\ketcalcdepth[0]{C}|
\itemket{Description}These functions return the size of C using current unit to the counter \verb|ketpicctr1|.
If option is 1, it displays the value.
\itemket{}\verb|\ketcalcwidth[0]{C}| returns the width of C.
\itemket{}\verb|\ketcalcheight[0]{C}| returns the height of C.
\itemket{}\verb|\ketcalcdepth[0]{C}| returns the depth of C.
\itemket{Examples} \verb|\ketcalcwidth[0]{abc}, \theketpicctra, \ketcalcwidth[1]{abc}| \par
It displays ``\ketcalcwidth[0]{abc}, \theketpicctra, \ketcalcwidth[1]{abc}''.
\end{cmd}

%------------ ketcalcwh ------------
\begin{cmd}{\bs ketcalcwh}{ketcalcwh}
\itemket{Usage}\verb|\ketcalcwh{C}|
\itemket{Description}This function displays the width and height of C using \verb|mm| in the form \{width\}\{height\}.
\itemket{Examples} \verb|\ketcalcwh{abc}|\par
It displays ``\ketcalcwh{abc}''.
\end{cmd}

%------------ dangerbendmark ------------
\begin{cmd}{\bs dangerbendmark}{dangerbendmark}
\itemket{Usage}\verb|\dangerbendmark[size]|
\itemket{Description}This function displays the symbol ``Dangerous turning point'' of Bulbaki.
\itemket{Examples} \verb|\dangerbendmark[1.2]| --->\ \ \dangerbendmark[1.2]
\end{cmd}

%------------ cautionmark ------------
\begin{cmd}{\bs cautionmark}{cautionmark}
\itemket{Usage}\verb|\cautionmark[size]|
\itemket{Description}This function displays the caution mark.
\itemket{Examples} \verb|\caoutionmark[1.2]| --->\ \ \cautionmark[1.2]
\end{cmd}

%------------ circlemark ------------
\begin{cmd}{\bs circlemark}{circlemark}
\itemket{Usage}\verb|\circlemark[thickness]{size}|
\itemket{Description}This function displays the circle.
If size=1, the diameter of the circle is 4mm.
\itemket{Examples} \verb|\circlemark[8]{1.2}| --->\ \ \circlemark[8]{1.2}
\end{cmd}

%------------ circleshade ------------
\begin{cmd}{\bs circleshade}{circleshade}
\itemket{Usage}\verb|\circleshade[thickness]{size}{density}|
\itemket{Description}This function displays the solid circle.
\itemket{Examples} \verb|\circleshade[8]{1.2}{0.3}| --->\ \ \circleshade[8]{1.2}{0.3}
\end{cmd}

%------------ arrow of increase or decrease ------------
\begin{cmd}{\bs NEarrow, \bs NELarrow, ...}{arrow of i or d}
\itemket{Usage}\verb|\NEarrow[size]|, \verb|\NELarrow[size}|, \verb|\NERarrow[size]|, 
\itemket{Description}These functions display the arrow of increase or decrease.
\itemket{Examples} \mbox{}

\vspace{-3mm}\hspace{16mm}
\begin{tabular}{|rl|rl|rl|rl|}
\hline
 \verb|\NEarrow| & \NEarrow &  \verb|\SEarrow| & \SEarrow & \verb|\NWarrow| & \NWarrow & \verb|\SWarrow| & \SWarrow \\
\hline
\verb|\NELarrow| & \NELarrow & \verb|\SELarrow| & \SELarrow & \verb|\NWLarrow| & \NWLarrow & \verb|\SWLarrow| & \SWLarrow \\
\hline
\verb|\NERarrow| & \NERarrow & \verb|\SERarrow| & \SERarrow & \verb|\NWRarrow| & \NWRarrow & \verb|\SWRarrow| & \SWRarrow \\
\hline
\end{tabular}
\end{cmd}

%=-=-=-=-= Macros of ketlayer =-=-=-=-=
\subsection{Macros of ketlayer}
Macros of ketlayer are used in layer environment.

Some macros take the form of connected main part and direction (``c'', ``e'', ``w'', ``s'', ``n'').
In the following we write them as ``main part + dir''.
Direction can be combine like as options of {\ketcindy} commands.
% (``ne'', ``nw'', ``se'', ``sw'')

For example, if main part is ``\verb|putnote|'', ``\verb|putnote+dir|'' are \par
``\verb|putnotec|'', ``\verb|putnotee|'', ``\verb|putnotew|'', ``\verb|putnotes|'', ``\verb|putnoten|'', ``\verb|putnotene|'', ``\verb|putnotenw|'', ``\verb|putnotese|'', ``\verb|putnotesw|''.

%-------------putnote+dir-------------------------------
\begin{cmd}{\bs putnote+dir}{putnote}
\itemket{Usage}\verb|\putnote+dir{x}{y}{Char}|
\itemket{Description}These functions put Char in the direction of dir of coordinates (x, y).
\itemket{}\verb|putnotec{x}{y}{Char}| puts Char with (x,y) as the center.
\itemket{}\verb|putnotee{x}{y}{Char}| puts Char on the right of (x,y).
\itemket{}\verb|putnotew{x}{y}{Char}| puts Char on the left of (x,y).
\itemket{}\verb|putnotes{x}{y}{Char}| puts Char under (x,y).
\itemket{}\verb|putnoten{x}{y}{Char}| puts Char above (x,y).
\itemket{Example}\mbox{}\\
\verb|\putnotese{20}{10}{\fbox{$\dfrac{1}{2}$}}|\\
\verb|\putnotec{40}{10}{\fbox{$\dfrac{1}{3}$}}|\\

\begin{layer}{60}{30}
\putnotese{20}{10}{\fbox{$\dfrac{1}{2}$}}
\putnotec{40}{10}{\fbox{$\dfrac{1}{3}$}}
\end{layer}
\vspace{30mm}
\end{cmd}

%-------------boxframe+dir-------------------------------
\begin{cmd}{\bs boxframe+dir}{boxframe}
\itemket{Usage}\verb|\boxframe+dir[thickness]{x}{y}{W}{H}{Strings}|
\itemket{Description}These functions draw a rectangle with width W and height H in the direction of dir of coordinates (x, y), and put the strings inside.
\end{cmd}

%-------------dashboxframe+dir-------------------------------
\begin{cmd}{\bs dashboxframe+dir}{dashboxframe}
\itemket{Usage}\verb|\dashboxframe+dir[thickness]{x}{y}{W}{H}{Strings}|
\itemket{Description}These functions draw a dashed rectangle with width W and height H in the direction of dir of coordinates (x, y), and put the strings inside.
\end{cmd}

%-------------jaggyboxframe+dir-------------------------------
\begin{cmd}{\bs jaggyboxframe+dir}{jaggyboxframe}
\itemket{Usage}\verb|\jaggyboxframe+dir[thickness]{x}{y}{W}{H}{Strings}|
\itemket{Description}These functions draw a jaggy rectangle with width W and height H in the direction of dir of coordinates (x, y), and put the strings inside.
\end{cmd}

%-------------diaboxframe+dir-------------------------------
\begin{cmd}{\bs diaboxframe+dir}{diaboxframe}
\itemket{Usage}\verb|\diaboxframe+dir[thickness]{x}{y}{W}{H}{Strings}|
\itemket{Description}These functions draw a rectangle with width W, height H, connecting diamond shapes, in the direction of dir of coordinates (x, y), and put the strings inside.
\end{cmd}

%-------------eraser+dir-------------------------------
\begin{cmd}{\bs eraser+dir}{eraser}
\itemket{Usage}\verb|\eraser+dir[F]{x}{y}{W}{H}|
\itemket{Description}These functions erase the interior of rectangle with width W and height H in the direction of dir of coordinates (x, y).
If F=0, it don't draw border lines. By default, F=1.
\end{cmd}

%-------------shadebox+dir-------------------------------
\begin{cmd}{\bs shadebox+dir}{shadebox}
\itemket{Usage}\verb|\shadebox+dir[F]{x}{y}{W}{H}{C1}{C2}|
\itemket{Description}These functions draw a rectangle with width W and height H in the direction of dir of coordinates (x, y), paint inside with color C1, and draw a border with color C2.
If F=0, they don't draw border lines. By default, F=0.

\end{cmd}

%-------------example 1-------------------------------
\vspace{5mm}

\begin{layer}{160}{0}
     \boxframese{000}{0}{30}{16}{boxframe}
 \dashboxframese{035}{0}{30}{16}{dashboxframe}
\jaggyboxframese{070}{0}{30}{16}{jaggyboxframe}
  \diaboxframese{105}{0}{30}{16}{diaboxframe}
  \shadeboxse[0]{140}{0}{30}{16}{yellow}{black}
\end{layer}

\vspace{15mm}
\newpage

%-------------popframe-------------------------------
\begin{cmd}{\bs popframe}{popframe}
\itemket{Usage}\verb|\popframe[thickness]{x}{y}{Dummy}{Cs}{Dummy}{Cp}{Cf}{Strings}|
\itemket{Description}This function draws a rectangle on the lower right (se) of the coordinates \text{(x, y)}, put strings inside and add a shadow of the color Cs.
\itemket{Details}Cp is background color. Cf is border color.\par
\Ltab{18.5mm}{}{\tt Note.\ }Dummy(color name) are currently ignored.\par
\Ltab{18.5mm}{}The size of the rectangle is determined automatically from strings.\par
\Ltab{18.5mm}{}The line thickness is 8 by default.\par
\Ltab{18.5mm}{}Strings must be width$\leq$ 200 mm, height$\leq$ 100 mm.
\end{cmd}

%-------------colorframe-------------------------------
\begin{cmd}{\bs colorframe}{colorframe}
\itemket{Usage}\verb|\colorframe[thickness]{x}{y}{Cp}{Cs}{Cf}{Strings}|
\itemket{Description}This function draws a rectangle on the lower right (se) of the coordinates \text{(x, y)}, put strings inside.
\itemket{Details}Cp is background color. Cf is border color.\par
\Ltab{18.5mm}{}{\tt Note.\ }Dummy(color name) is ignored.\par
\Ltab{18.5mm}{}The size of the rectangle is determined automatically from strings.\par
\Ltab{18.5mm}{}The line thickness is 8 by default.\par
\Ltab{18.5mm}{}Strings must be width$\leq$ 200 mm, height$\leq$ 100 mm.
\end{cmd}

%-------------example 2-------------------------------
\vspace{15mm}
\noindent{\bf Examples.}\par
\verb|\definecolor{shade}{cmyk}{0,0,0,0.4}| $\gets$ color name ``shade'' defined.\par
\verb|\popframe[16]{40}{5}{white}{shade}{white}{cyan}{red}{\Large\tt POP frame}| \par
\verb|\colorframe[16]{90}{5}{yellow}{white}{blue}{\Large\tt COLOR frame}| \\

\begin{layer}{160}{0}
\definecolor{shade}{cmyk}{0,0,0,0.4}
%  \popframe[16]{40}{5}{Ds}{shade}{Dp}{yellow}{green}{\Large\tt POP frame}
  \popframe[16]{40}{5}{Ds}{shade}{Dp}{cyan}{red}{\Large\tt POP frame}
\colorframe[16]{90}{5}{yellow}{green}{blue}{\Large\tt COLOR frame}
\end{layer}

\vspace{25mm}

%-------------cirscoremark-------------------------------
\begin{cmd}{\bs cirscoremark}{cirscoremark}
\itemket{Usage}\verb|\cirscoremark[thickness]{size}|
\itemket{Description}This function draws a handwritten double circle.
\end{cmd}

%-------------scirscoremark-------------------------------
\begin{cmd}{\bs scirscoremark}{scirscoremark}
\itemket{Usage}\verb|\scirscoremark[thickness]{size}|
\itemket{Description}This function draws a handwritten single circle.
\end{cmd}

%-------------triscoremark-------------------------------
\begin{cmd}{\bs triscoremark}{triscoremark}
\itemket{Usage}\verb|\triscoremark[thickness]{size}|
\itemket{Description}This function draws a handwritten triangle.
\end{cmd}

%-------------crosscoremark-------------------------------
\begin{cmd}{\bs crosscoremark}{crosscoremark}
\itemket{Usage}\verb|\crosscoremark[thickness]{size}|
\itemket{Description}This function draws a handwritten cross mark.
\end{cmd}

\begin{layer}{170}{0}
\putnotec{30}{25}{\cirscoremark{0.8}}
\putnotec{60}{25}{\scirscoremark{0.8}}
\putnotec{90}{25}{\triscoremark{0.8}}
\putnotec{120}{25}{\crosscoremark{0.8}}
\end{layer}

\vspace{45mm}

%-------------lineseg-------------------------------
\begin{cmd}{\bs lineseg}{lineseg}
\itemket{Usage}\verb|\lineseg[thickness]{x}{y}{L}{|{$\theta$}\verb|}|
\itemket{Description}This function draws a line segment of length L from the coordinates (x, y) in the direction of $\theta^\circ$ degrees. 
\itemket{Details}Unit of length L is mm.\par
\Ltab{18.5mm}{}The line thickness is 12 by default. Unit is milli inch\par
\Ltab{18.5mm}{}x, y, $\theta$ may be decimal.
\itemket{Example}\verb|\lineseg[16]{135}{25}{30}{25}|

\begin{layer}{160}{0}
\lineseg[16]{60}{20}{30}{25}
%\arrowlineseg[16]{130}{50}{10}{45}
\end{layer}

\vspace{25mm}
\end{cmd}
%-------------dashlineseg-------------------------------
\begin{cmd}{\bs dashlineseg}{dashlineseg}
\itemket{Usage}\verb|\dashlineseg[thickness]{x}{y}{L}{|{$\theta$}\verb|}|
\itemket{Description}This function draws a dash line segment of length L from the coordinates \text{(x, y)} in the direction of $\theta^\circ$ degrees. 
\itemket{Details}Unit of length L is mm.\par
\Ltab{18.5mm}{}The line thickness is 12 by default. Unit is milli inch\par
\Ltab{18.5mm}{}x, y, $\theta$ may be decimal.

\end{cmd}
%-------------arrowlineseg-------------------------------
\begin{cmd}{\bs arrowlineseg}{arrowlineseg}
\itemket{Usage}\verb|\arrowlineseg[thickness]{x}{y}{L}{|{$\theta$}\verb|}|
\itemket{Description}This function draws a arrow line segment of length L from the coordinates \text{(x, y)} in the direction of $\theta^\circ$ degrees. 
\itemket{Details}The arrowhead is drawn at the starting point.\par
\Ltab{18.5mm}{}The line thickness is 12 by default. Unit is milli inch.\par
\Ltab{18.5mm}{}x, y, $\theta$ may be decimal.
\itemket{Example}\verb|\arrowlineseg[16]{60}{20}{10}{45}|

\begin{layer}{160}{0}
%\lineseg[16]{60}{20}{30}{25}
\arrowlineseg[16]{60}{15}{10}{45}
\end{layer}

\vspace{20mm}
\end{cmd}

%-------------arrowhead-------------------------------
\begin{cmd}{\bs arrowhead}{arrowhead}
\itemket{Usage}\verb|\arrowhead[size]{x}{y}{|{$\theta$}\verb|}|
\itemket{Description}This function draws a arrowhead on the coordinates (x, y) in the direction of $\theta^\circ$ degrees. 
\itemket{Details}The line thickness is 12 by default. Unit is milli inch.\par
\Ltab{18.5mm}{}x, y, $\theta$ may be decimal.
\end{cmd}

%-------------hjaggyline-------------------------------
\begin{cmd}{\bs hjaggyline}{hjaggyline}
\itemket{Usage}\verb|\hjaggyline[thickness]{x}{y}{W}|
\itemket{Description}This function draws a jagged line of length W from the coordinates (x, y) to the right. 
\end{cmd}

%-------------hjaggylineb-------------------------------
\begin{cmd}{\bs hjaggylineb}{hjaggylineb}
\itemket{Usage}\verb|\hjaggylineb[thickness]{x}{y}{W}|
\itemket{Description}This function draws a jagged line of length W from the coordinates (x, y) to the right. 
\itemket{Details}This function draws a reverse jagged line against ``hjaggyline''.\par
\end{cmd}

%-------------vjaggyline-------------------------------
\begin{cmd}{\bs vjaggyline}{vjaggyline}
\itemket{Usage}\verb|\vjaggyline[thickness]{x}{y}{W}|
\itemket{Description}This function draws a jagged line of length W from the coordinates (x, y) to the right. 
\end{cmd}

%-------------vjaggylineb-------------------------------
\begin{cmd}{\bs vjaggylineb}{vjaggylineb}
\itemket{Usage}\verb|\vjaggylineb[thickness]{x}{y}{W}|
\itemket{Description}This function draws a jagged line of length W from the coordinates (x, y) to the right. 
\itemket{Details}This function draws a reverse jagged line against ``vjaggyline''.\par
\end{cmd}

%-------------example 3-------------------------------
\noindent{\bf Examples.}\par
\Ltab{18.5mm}{}\verb|\hjaggyline{40}{10}{15}| \par
\Ltab{18.5mm}{}\verb|\hjaggylineb{40}{20}{15}| \par
\Ltab{18.5mm}{}\verb|\vjaggyline{70}{10}{15}| \par
\Ltab{18.5mm}{}\verb|\vjaggylineb{90}{10}{15}| \par

\begin{layer}{160}{0}
 \hjaggyline{40}{10}{15}
\hjaggylineb{40}{20}{15}
 \vjaggyline{70}{10}{15}
\vjaggylineb{90}{10}{15}
\end{layer}

\vspace{35mm}

%-------------circleline-------------------------------
\begin{cmd}{\bs circleline}{circleline}
\itemket{Usage}\verb|\circleline{x}{y}{size}|
\itemket{Description}This function draws a circle with (x, y) as the center.
\end{cmd}

%-------------ballonr-------------------------------
\begin{cmd}{\bs ballonr}{ballonr}
\itemket{Usage}\verb|\ballonr[thickness]{x}{y}{size}{Char}|
\itemket{Description}This function draws a balloon in the upper right side from (x, y) and, puts Char inside.
\end{cmd}

%-------------ballonl-------------------------------
\begin{cmd}{\bs ballonl}{ballonl}
\itemket{Usage}\verb|\ballonl[thickness]{x}{y}{size}{Char}|
\itemket{Description}This function draws a balloon in the upper left side from (x, y) and, puts Char inside.
\end{cmd}

%-------------lefthand-------------------------------
\begin{cmd}{\bs lefthand}{lefthand}
\itemket{Usage}\verb|\lefthand[thickness]{x}{y}|
\itemket{Description}This function draws a fingertip on (x, y).
\end{cmd}

%-------------righthand-------------------------------
\begin{cmd}{\bs righthand}{righthand}
\itemket{Usage}\verb|\righthand[thickness]{x}{y}|
\itemket{Description}This function draws a fingertip on (x, y).
\end{cmd}

%-------------leftdownhand-------------------------------
\begin{cmd}{\bs leftdownhand}{leftdownhand}
\itemket{Usage}\verb|\leftdownhand[thickness]{x}{y}|
\itemket{Description}This function draws a fingertip on (x, y).
\end{cmd}

%-------------rightdownhand-------------------------------
\begin{cmd}{\bs rightdwonhand}{rightdownhand}
\itemket{Usage}\verb|\rightdownhand[thickness]{x}{y}|
\itemket{Description}This function draws a fingertip on (x, y).
\end{cmd}

%-------------example 3-------------------------------
\noindent{\bf Examples.}\par

\begin{layer}{170}{0}
      \ballonr{30}{35}{1}{Example1}
      \ballonl{90}{30}{1}{Example2}
     \lefthand{120}{25}
    \righthand{140}{25}
 \leftdownhand{120}{15}
\rightdownhand{140}{15}
\end{layer}

\newpage

%-==Command List ========================
\hypertarget{functionlist}{}
\section{Command List}
%\hyperlink{index}{To index}

\begin{tabbing}
12345678901234567890\=\kill

{\bf Macros of ketpic} \> \\
\hyperlink{ketpic}{\bs ketpic} \> logo of \ketpic\\
\hyperlink{ketcindy}{\bs ketcindy} \> logo of \ketcindy\\
\hyperlink{tab}{\bs Ltab} \> left tab\\
\hyperlink{tab}{\bs Rtab} \> right tab\\
\hyperlink{tab}{\bs Ctab} \> center tab\\
\hyperlink{ketcalc}{\bs ketcalcwidth} \> returns the width of strings\\
\hyperlink{ketcalc}{\bs ketcalcheight} \> returns the height of strings\\
\hyperlink{ketcalc}{\bs ketcalcdepth} \> returns the depth of strings\\
\hyperlink{ketcalcwh}{\bs ketcalcwh} \> returns the width and height of strings\\
\hyperlink{dangerbendmark}{\bs dangerbendmark} \> symbol ``Dangerous turning point'' of Bulbaki\\
\hyperlink{cautionmark}{\bs cautionmark} \> caution mark\\
\hyperlink{circlemark}{\bs circlemark} \> circle\\
\hyperlink{circleshade}{\bs circleshade} \> solid circle\\
\hyperlink{arrow of i or d}{\bs NEarrow, ...} \> arrow of increase or decrease\\

{\bf Macros of ketlayer} \> \\
\hyperlink{putnote}{\bs putnote+dir} \> puts Char\\
\hyperlink{boxframe}{\bs boxframe+dir} \> draws a rectangle and puts strings\\
\hyperlink{dashboxframe}{\bs dashboxframe+dir} \> draws a dashed rectangle and puts strings\\
\hyperlink{jaggyboxframe}{\bs jaggyboxframe+dir} \> draws a jaggy rectangle and puts strings\\
\hyperlink{diaboxframe}{\bs diaboxframe+dir} \> draws a diamond chaining rectangle and puts strings\\
\hyperlink{eraser}{\bs eraser+dir} \> erases the interior of a rectangle\\
\hyperlink{shadebox}{\bs shadebox+dir} \> draws a shaded rectangle and puts strings\\
\hyperlink{popframe}{\bs popframe} \> draws a rectangle and shade with the specified color and puts strings\\
\hyperlink{colorframe}{\bs colorframe} \> draws a rectangle with the specified color and puts strings\\
\hyperlink{cirscoremark}{\bs cirscoremark} \> draws a handwritten double circle\\
\hyperlink{scirscoremark}{\bs scirscoremark} \> draws a handwritten single circle\\
\hyperlink{triscoremark}{\bs triscoremark} \> draws a handwritten triangle\\
\hyperlink{crosscoremark}{\bs crosscoremark} \> draws a handwritten cross mark\\
\hyperlink{lineseg}{\bs lineseg} \> draws a line segment specified angle\\
\hyperlink{dashlineseg}{\bs dashlineseg} \> draws a dashed line segment specified angle\\
\hyperlink{arrowlineseg}{\bs arrowlineseg} \> draws a arrow line segment specified angle\\
\hyperlink{arrowhead}{\bs arrowhead} \> draws a arrowhead specified angle\\
\hyperlink{hjaggyline}{\bs hjaggyline} \> draws a horizontal jaggy line segment\\
\hyperlink{hjaggylineb}{\bs hjaggylineb} \> draws a horizontal jaggy line segment against \bs hjaggyline\\
\hyperlink{vjaggyline}{\bs vjaggyline} \> draws a vertical jaggy line segment\\
\hyperlink{vjaggylineb}{\bs vjaggylineb} \> draws a vertical jaggy line segment against \bs vjaggyline\\
\hyperlink{circleline}{\bs circleline} \> draws a circle\\
\hyperlink{ballonl}{\bs ballonl} \> draws a ballon and puts strings inside\\
\hyperlink{ballonr}{\bs ballonr} \> draws a ballon and puts strings inside\\
%\hyperlink{lefthand}{\bs lefthand} \> draws a leftward fingertip\\
%\hyperlink{righthand}{\bs righthand} \> draws rightward fingertip\\
\hyperlink{lefthand}{\bs lefthand} \> draws fingertip\\
\hyperlink{righthand}{\bs righthand} \> draws fingertip\\
\hyperlink{leftdownhand}{\bs leftdownhand} \> draws fingertip\\
\hyperlink{rightdownhand}{\bs rightdownhand} \> draws fingertip\\

\end{tabbing}

\end{document}