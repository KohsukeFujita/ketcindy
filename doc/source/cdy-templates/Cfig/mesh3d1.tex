%%% /Users/hannya/Desktop/fig/table.tex 
%%% Generator=table.cdy 
{\unitlength=1cm%
\begin{picture}%
(13,8.4)(0,0)%
\special{pn 8}%
%
\special{pa     0 -3307}\special{pa     0    -0}%
\special{fp}%
\special{pa  1181 -3307}\special{pa  1181    -0}%
\special{fp}%
\special{pa  5118 -3307}\special{pa  5118    -0}%
\special{fp}%
\special{pa     0 -3307}\special{pa  5118 -3307}%
\special{fp}%
\special{pa     0 -2992}\special{pa  5118 -2992}%
\special{fp}%
\special{pa     0 -2362}\special{pa  5118 -2362}%
\special{fp}%
\special{pa     0 -1575}\special{pa  5118 -1575}%
\special{fp}%
\special{pa     0  -787}\special{pa  5118  -787}%
\special{fp}%
\special{pa     0    -0}\special{pa  5118    -0}%
\special{fp}%
\settowidth{\Width}{topology}\setlength{\Width}{-0.5\Width}%
\settoheight{\Height}{topology}\settodepth{\Depth}{topology}\setlength{\Height}{-0.5\Height}\setlength{\Depth}{0.5\Depth}\addtolength{\Height}{\Depth}%
\put(  1.500,  8.000){\hspace*{\Width}\raisebox{\Height}{topology}}%
%
\settowidth{\Width}{説明}\setlength{\Width}{-0.5\Width}%
\settoheight{\Height}{説明}\settodepth{\Depth}{説明}\setlength{\Height}{-0.5\Height}\setlength{\Depth}{0.5\Depth}\addtolength{\Height}{\Depth}%
\put(  8.000,  8.000){\hspace*{\Width}\raisebox{\Height}{説明}}%
%
\settowidth{\Width}{open}\setlength{\Width}{-0.5\Width}%
\settoheight{\Height}{open}\settodepth{\Depth}{open}\setlength{\Height}{-0.5\Height}\setlength{\Depth}{0.5\Depth}\addtolength{\Height}{\Depth}%
\put(  1.500,  6.800){\hspace*{\Width}\raisebox{\Height}{open}}%
%
\settowidth{\Width}{ \begin{minipage}{95mm}端点もしくは面の端の点までが面になる。その結果、(m-1)×(n-1)個の矩形ができる。 面は両サイドと1つの境界を持つ。 \end{minipage}}\setlength{\Width}{0\Width}%
\settoheight{\Height}{ \begin{minipage}{95mm}端点もしくは面の端の点までが面になる。その結果、(m-1)×(n-1)個の矩形ができる。 面は両サイドと1つの境界を持つ。 \end{minipage}}\settodepth{\Depth}{ \begin{minipage}{95mm}端点もしくは面の端の点までが面になる。その結果、(m-1)×(n-1)個の矩形ができる。 面は両サイドと1つの境界を持つ。 \end{minipage}}\setlength{\Height}{-0.5\Height}\setlength{\Depth}{0.5\Depth}\addtolength{\Height}{\Depth}%
\put(  3.050,  6.800){\hspace*{\Width}\raisebox{\Height}{ \begin{minipage}{95mm}端点もしくは面の端の点までが面になる。その結果、(m-1)×(n-1)個の矩形ができる。 面は両サイドと1つの境界を持つ。 \end{minipage}}}%
%
\settowidth{\Width}{closerows}\setlength{\Width}{-0.5\Width}%
\settoheight{\Height}{closerows}\settodepth{\Depth}{closerows}\setlength{\Height}{-0.5\Height}\setlength{\Depth}{0.5\Depth}\addtolength{\Height}{\Depth}%
\put(  1.500,  5.000){\hspace*{\Width}\raisebox{\Height}{closerows}}%
%
\settowidth{\Width}{ \begin{minipage}{95mm}各行の最初と最後の頂点が結合されて対応する面ができる。その結果、(m-1)×n 個の矩形ができる。面は両サイドと2つの境界を持つ。 \end{minipage}}\setlength{\Width}{0\Width}%
\settoheight{\Height}{ \begin{minipage}{95mm}各行の最初と最後の頂点が結合されて対応する面ができる。その結果、(m-1)×n 個の矩形ができる。面は両サイドと2つの境界を持つ。 \end{minipage}}\settodepth{\Depth}{ \begin{minipage}{95mm}各行の最初と最後の頂点が結合されて対応する面ができる。その結果、(m-1)×n 個の矩形ができる。面は両サイドと2つの境界を持つ。 \end{minipage}}\setlength{\Height}{-0.5\Height}\setlength{\Depth}{0.5\Depth}\addtolength{\Height}{\Depth}%
\put(  3.050,  5.000){\hspace*{\Width}\raisebox{\Height}{ \begin{minipage}{95mm}各行の最初と最後の頂点が結合されて対応する面ができる。その結果、(m-1)×n 個の矩形ができる。面は両サイドと2つの境界を持つ。 \end{minipage}}}%
%
\settowidth{\Width}{closecolumns}\setlength{\Width}{-0.5\Width}%
\settoheight{\Height}{closecolumns}\settodepth{\Depth}{closecolumns}\setlength{\Height}{-0.5\Height}\setlength{\Depth}{0.5\Depth}\addtolength{\Height}{\Depth}%
\put(  1.500,  3.000){\hspace*{\Width}\raisebox{\Height}{closecolumns}}%
%
\settowidth{\Width}{ \begin{minipage}{95mm}各列の最初と最後の頂点が結合されて対応する面ができる。その結果、m×(n-1) 個の矩形ができる。面は両サイドと2つの境界を持つ。  \end{minipage}}\setlength{\Width}{0\Width}%
\settoheight{\Height}{ \begin{minipage}{95mm}各列の最初と最後の頂点が結合されて対応する面ができる。その結果、m×(n-1) 個の矩形ができる。面は両サイドと2つの境界を持つ。  \end{minipage}}\settodepth{\Depth}{ \begin{minipage}{95mm}各列の最初と最後の頂点が結合されて対応する面ができる。その結果、m×(n-1) 個の矩形ができる。面は両サイドと2つの境界を持つ。  \end{minipage}}\setlength{\Height}{-0.5\Height}\setlength{\Depth}{0.5\Depth}\addtolength{\Height}{\Depth}%
\put(  3.050,  3.000){\hspace*{\Width}\raisebox{\Height}{ \begin{minipage}{95mm}各列の最初と最後の頂点が結合されて対応する面ができる。その結果、m×(n-1) 個の矩形ができる。面は両サイドと2つの境界を持つ。  \end{minipage}}}%
%
\settowidth{\Width}{closeboth}\setlength{\Width}{-0.5\Width}%
\settoheight{\Height}{closeboth}\settodepth{\Depth}{closeboth}\setlength{\Height}{-0.5\Height}\setlength{\Depth}{0.5\Depth}\addtolength{\Height}{\Depth}%
\put(  1.500,  1.000){\hspace*{\Width}\raisebox{\Height}{closeboth}}%
%
\settowidth{\Width}{ \begin{minipage}{95mm}各行の最初と最後および各列の最初と最後の頂点が結合されて対応する面ができる。その結果、m×n 個の矩形ができる。面は両サイドを持ち境界はない。 \end{minipage}}\setlength{\Width}{0\Width}%
\settoheight{\Height}{ \begin{minipage}{95mm}各行の最初と最後および各列の最初と最後の頂点が結合されて対応する面ができる。その結果、m×n 個の矩形ができる。面は両サイドを持ち境界はない。 \end{minipage}}\settodepth{\Depth}{ \begin{minipage}{95mm}各行の最初と最後および各列の最初と最後の頂点が結合されて対応する面ができる。その結果、m×n 個の矩形ができる。面は両サイドを持ち境界はない。 \end{minipage}}\setlength{\Height}{-0.5\Height}\setlength{\Depth}{0.5\Depth}\addtolength{\Height}{\Depth}%
\put(  3.050,  1.000){\hspace*{\Width}\raisebox{\Height}{ \begin{minipage}{95mm}各行の最初と最後および各列の最初と最後の頂点が結合されて対応する面ができる。その結果、m×n 個の矩形ができる。面は両サイドを持ち境界はない。 \end{minipage}}}%
%
\end{picture}}%