\documentclass{ujarticle}
\usepackage{ketpic,ketlayer}
\usepackage{amsmath,amssymb}
\usepackage{graphicx}
\usepackage{xcolor}
\usepackage{bm,enumerate}
\usepackage[dvipdfmx,colorlinks=true,urlcolor=blue]{hyperref}

\setmargin{20}{20}{20}{20}

\西暦

\renewcommand{\labelitemi}{・}
\renewcommand{\labelitemii}{・}

\begin{document}

\begin{center}
\ketcindy\ のインストール (Linux)
\end{center}

\hfill 修正日:\today

\begin{enumerate}[\bf\large 1.]
\item Linux (Ubnutu 20.04 LTS) へのインストール手順

\begin{enumerate}[(1)]

\item 作業用フォルダを作成して移動する

※ 作業用フォルダは \textasciitilde/tmp とする

\begin{verbatim}
$ cd
$ mkdir tmp
$ cd tmp
\end{verbatim}

\item TexLive 2020 のインストール

※ デフォルトのフォルダ( /usr/local/texlive/2020/ )にインストールする

\begin{verbatim}
$ wget http://mirror.ctan.org/systems/texlive/tlnet/install-tl-unx.tar.gz
$ tar xvfz install-tl-unx.tar.gz
$ sudo ./install-tl-(日時)/install-tl
\end{verbatim}

※ 選択肢が表示されたら I + Enter を入力

\verb|$ sudo /usr/local/texlive/2020/bin/x86_64-linux/tlmgr path add|

\item JDK のインストール

\verb|$ sudo apt install default-jdk|

\item Rのインストール

\verb|$ sudo apt install r-base-core|

\item git のインストール

\verb|$ sudo apt install git|

\item Cinderella3 のインストール

※ /opt/cinderella/ にインストールする

\begin{verbatim}
$ wget https://beta.cinderella.de/Cinderella-3.0b.1952.tar.gz
$ tar xvfz Cinderella-3.0b.1952.tar.gz
$ sudo mv cinderella/ /opt
\end{verbatim}

\item Github から最新版をクローンして \verb|ketcindysettings.cdy| を開く
\begin{verbatim}
$ git clone https://github.com/ketpic/ketcindy.git
$ /opt/cinderella/Cinderella ~/tmp/ketcindy/doc/ketcindysettings.cdy &
\end{verbatim}

\end{enumerate}


\item TeXworksを設定する.
  \begin{itemize}
  \item \url{https://github.com/TeXworks/texworks/releases/} からダウンロードできる.
  \end{itemize}

\item gccのインストール
  \begin{itemize}
    \item 曲面描画のためには, gccが必要である.
  \end{itemize}
\end{enumerate}

\end{document}
