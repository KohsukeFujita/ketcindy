\documentclass{ujarticle}
\usepackage{ketpic,ketlayer}
\usepackage{amsmath,amssymb}
\usepackage{graphicx}
\usepackage{xcolor}
\usepackage{bm,enumerate}
\usepackage[dvipdfmx,colorlinks=true,urlcolor=blue]{hyperref}

\setmargin{20}{15}{15}{15}

\西暦

\renewcommand{\labelitemi}{・}
\pagestyle{empty}

\begin{document}

\begin{center}
Emathのインストール
\end{center}

\vspace{-5mm}

\hfill 修正日:\today

\begin{enumerate}[\bf\large 1.]
\item emathのページ\url{http://emath.s40.xrea.com/index1.html}から\\
\hspace*{4zw}\verb|emathf051107c.zip (3,012,060 bytes)|\\
をダウンロードして解凍する.
\item さらに,sty.zipをその中に解凍する.
\item \verb|copyemath.cdy|を入れる.

\item[]\hspace*{1zw}注)1つのフォルダの中にcopyemath.cdyとフォルダstyがあるようにする.

%\item[]\hspace*{1zw}注)以下のURLからDLすると,1,2,3をパックにしたフォルダが得られる.\\
%\hspace*{4zw}\url{https://drive.google.com/file/d/1e06TQtWpPFqwg3JrwI3AOXdKYtIKE-ze/view?usp=sharing}

\item \verb|copyemath.cdy|を立ち上げて,TeXシステムを選び,\verb|CopyEmath|ボタンを押す.
\end{enumerate}

\end{document}