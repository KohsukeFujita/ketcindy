\documentclass{ujarticle}
\usepackage{ketpic,ketlayer}
\usepackage{amsmath,amssymb}
\usepackage{graphicx}
\usepackage{xcolor}
\usepackage{bm,enumerate}
\usepackage[dvipdfmx,colorlinks=true,urlcolor=blue]{hyperref}

\setmargin{15}{15}{15}{15}

\西暦

\renewcommand{\labelitemi}{・}
\pagestyle{empty}

\begin{document}

\begin{center}
emath(selected)のインストール
\end{center}

\vspace{-5mm}

\hfill 修正日:\today

\begin{enumerate}[\bf\large 1.]
\item \TeX システムの中にemathをおくフォルダを用意する.
   \begin{itemize}
   \item texlive, kettexの場合,例えば次のフォルダを作る.\\
  \verb|texmf-local > tex > platex|
   \item Macのkettex.appの場合,「フォルダへ移動」で次を入力してから上のフォルダを作る.\\
  \verb|/Applications/kettex.app/texlive|
   \end{itemize}

\item \verb|emathselected|フォルダを丸ごとコピーする.
\item ketcindysettings.cdyを立ち上げ,TeXシステムを選択して\verb|Update|だけを実行する.

\end{enumerate}

\end{document}