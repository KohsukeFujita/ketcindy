\documentclass{ujarticle}
\usepackage{ketpic,ketlayer}
\usepackage{amsmath,amssymb}
\usepackage{graphicx}
\usepackage{xcolor}
\usepackage{bm,enumerate}
\usepackage[dvipdfmx,colorlinks=true,urlcolor=blue]{hyperref}

\setmargin{20}{20}{20}{20}

\西暦

\renewcommand{\labelitemi}{・}
\renewcommand{\labelitemii}{・}

\begin{document}

\begin{center}
KETCindyについての補足(Mac)
\end{center}

\hfill 修正日:\today

\begin{enumerate}[\bf\large 1.]

\item KeTTeXのインストール

\begin{itemize}
\item 以下から\verb|kettex.dmg|をダウンロードする.\\
\hspace*{3mm}\url{https://drive.google.com/drive/folders/1h_HDcKSp3S6qarbTSiUn9U5brgOGbU93?usp=sharing}
\item kettex.dmgをダブルクリックして仮想ディスクを開き,kettexをApplicationsに入れる.\\
\hspace*{1zw}注)仮想ディスクにあるApplicationsにドラグドロップすればよい.
\item
最初の実行時,セキュリティの制限によって実行不可のエラーが出る場合(特にCatalina以降)
    \begin{enumerate}[(1)]
   \item キャンセルを押す.(ゴミ箱を選ばない)
   \item システム環境設定>セキュリティとプライバシー を開く.
   \item 「このまま許可」をクリックする.
   \end{enumerate}

\end{itemize}

\item KeTCindyのインストール
\begin{itemize}
\item ketcindysettings.cdyを利用する.
\begin{enumerate}[(1)]
\item 必要なら,「情報を見る」でアプリをCinderella.appを選び,「すべてを変更」
\item 左方にあるボタンで,言語,TeXの種類,描画コードを選ぶ.\\
\hspace*{10mm}ボタンを押すと順に項目が変わる.
\item 中央にあるボタンでTeXシステムを選ぶ.\\
\hspace*{10mm}KeTTeX,TeXLive以外の場合は,CindyScriptでパスを設定してから.\verb|Other|を選ぶ.
\item 右側にあるボタンを順に押す.\\
\hspace*{10mm}\verb|Mkinit|:初期化ファイルketcindy.iniをユーザホームに作成\\
\hspace*{10mm}\verb|Update|:TeXに入っているketlib関連のファイルを更新(コピー)\\
\hspace*{10mm}\verb|Work|:作業フォルダketcindy.iniをユーザホームに作成\\
\hspace*{1mm}注)\verb|KeTTeX.app|を選択して\verb|Update|を実行した場合\\
\hspace*{13mm}\verb|/Applications|に\verb|KeTTeX.app/texlive|のシンボリックリンク\verb|kettexlive|が作られる.
\end{enumerate}
\item ketcindysettings.cdyのUpdateでエラーが出た場合\\
\hspace*{10mm}{\bf 1.}を実行する.
\end{itemize}

  \item KeTCindyのテストラン
    \begin{enumerate}[(1)]
    \item ketcindysettings.cdyを終了してから,testrun.cdyを開く.\\
      \hspace*{10mm}画面に白い枠が出れば,ライブラリの読み込みは成功.
    \item スクリーンの左上部にあるFigureボタンを押して,PDFが表示されれば成功.\\
\hspace*{10mm}KeTTeXでエラーが出た場合は{\bf 1.}を実行する.
\end{enumerate}
%\newpage

\item TeXWorksの設定(kettexの場合)
  \begin{itemize}
  \item \url{https://github.com/TeXworks/texworks/releases/} からダウンロードできる.\\
\hspace*{10mm}注)KeTTeXの場合,TeXworksのバージョンは 0.6.2の方がよい.
  \item TeXworksを立ち上げ,「TeXworks \verb|>| 環境設定 \verb|>| タイプセット」
  \item 上の欄(パス)に以下を選択して入れる\\
  \hspace*{5mm}\verb|/Applications/kettex/texlive/bin/x86_64-darwin|(kettexの場合)\\
  \hspace*{5mm}\verb|/Applications/kettexlive/bin/x86_64-darwin|(KeTTeX.appの場合.\verb|KeTTeX.app|が入る)\\
  \hspace*{10mm}注) この行を上の欄の先頭に移動する.
  \item 下の欄の横にある + をクリック
    \begin{itemize}
    \item 名前:uplatex(ptex2pdf)またはplatex(ptex2pdf)
    \item プログラム : ptex2pdf
    \item 引数:\\
    \hspace*{10mm} \verb|-u|(uplatexの場合のみ)\\
    \hspace*{10mm} \verb|-l|\\
    \hspace*{10mm} \verb|-ot|\\
    \hspace*{10mm}  \verb|$synctexoption|\\
    \hspace*{10mm}  \verb|$fullname|
    \item[]OKボタンを押し,デフォルトを変更してOKボタンを押す.
    \end{itemize}
  \end{itemize}

\item TeXShopの設定(kettexの場合)
  \begin{itemize}
  \item \verb|/Applications/TeX/TexShop.app|がなければ,以下からダウンロードする.\\
  \hspace*{5mm}\url{https://pages.uoregon.edu/koch/texshop/obtaining.html}
  \item TeXShopを立ち上げ,「TeXShop \verb|>| 環境設定 |」
  \item 「書類\verb|>|設定プロファイル」 ptex(ptex2pdf)かuptex(uptex2pdf)を選ぶ
  \item 「内部設定\verb|>|パス設定」以下を入れる.\\
  \hspace*{10mm}\verb|/Applications/kettex/texlive/bin/x86_64-darwin|(kettexの場合)\\
  \hspace*{10mm}\verb|/Applications/kettexlive/bin/x86_64-darwin|(KeTTeX.appの場合)
  \end{itemize}

\item gccのインストール
  \begin{itemize}
    \item 曲面描画のためには, gccが必要である.
    \item Xcodeがインストールされていなければ,インストールする.\\
    \hspace*{5mm}注) ターミナルで次を実行すれば,gccだけがインストールされる.\\
    \hspace*{20mm}\verb|sudo xcode-select --install|
  \end{itemize}

\item 手動でコピーする場合(KeTTeX)\\
\hspace*{1zw}注)他のTeXの場合は,適宜パスを置き換える.\\
\hspace*{3zw}\verb|/Applications/kettex/texlive| $=>$ \verb|/Library/TeX/Root| など
  \begin{enumerate}[(1)]
  \item ketcindy(-master)/ketcindyfolderを開いておく.
  \item scriptsフォルダの中身を以下にコピーする.\\
 \verb|/Applications/kettex/texlive/texmf-dist/scripts/ketcindy|
  \item styleフォルダの中身を以下にコピーする.\\
 \verb|/Applications/kettex/texlive/texmf-dist/tex/latex/ketcindy|
  \item docフォルダの中身を以下にコピーする.\\
 \verb|/Applications/kettex/texlive/texmf-dist/doc/support/ketcindy|
  \item ターミナルで以下を実行する\\
  \hspace*{1zw}\verb|sudo /Applications/kettex/texlive/bin/x86_64-darwin/mktexlsr|
  \item workをユーザホームなど適当な場所にコピーして,名前(例えばketcindy)を変更する.
  \item 上の作業ディレクトリ(ketcindy)に doc/ketmanual のマニュアルをコピーする.
  \end{enumerate}

\item その他

\begin{itemize}
 \item 「すべての実行を許可」を表示させる
\begin{enumerate}[(1)]
\item ターミナルで \verb|sudo stctl --master-disable| を実行
\item システム環境設定>セキュリティとプライバシー を開く
\item 実行の許可が「すべてを許可」になっているかを確認する
   \end{enumerate}

    \item PDFの表示後,ターミナル画面を閉じるようにする
       \begin{enumerate}[(1)]
        \item アプリケーション \verb|/| ユーティリティ \verb|/| ターミナルを開く
        \item トップメニューから\\
          \hspace*{5mm}ターミナル>環境設定 \verb|>|(プロファイル)\verb|>| シェル\\
          \hspace*{10mm}「シェルが正常に終了した場合閉じる」を選択
        \end{enumerate}
\end{itemize}

\end{enumerate}

\end{document}