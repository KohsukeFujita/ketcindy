\documentclass{ujarticle}
\usepackage{ketpic,ketlayer}
\usepackage{amsmath,amssymb}
\usepackage{graphicx}
\usepackage{xcolor}
\usepackage{bm,enumerate}
\usepackage[dvipdfmx,colorlinks=true,urlcolor=blue]{hyperref}

\setmargin{20}{20}{20}{20}

\西暦

\renewcommand{\labelitemi}{・}
\renewcommand{\labelitemii}{・}

\begin{document}

\begin{center}
KeTCindyについての補足 (Windows)
\end{center}

\hfill 修正日:\today

\begin{enumerate}[\bf\large 1.]

\item Cinderellaのインストール
\begin{itemize}
\item \url{https://beta.cinderella.de}から「保存」でダウンロードする.
\item インストーラを右クリック「管理者として実行」を選ぶ.
\item インストール先をProgram Files(または (x86))を選ぶ.
\end{itemize}

\item Sumatraのインストール
\begin{itemize}
\item \url{https://www.sumatrapdfreader.org/download-free-pdf-viewer.html} からダウンロードする.
\item {\color{red}\verb|Option|を選び},インストール先をProgram Files(または (x86))にする.
\end{itemize}

\item KeTTeXのインストール

\begin{enumerate}[(1)]
\item \verb|KeTTeX-windows-20210322.zip|を以下からダウンロードして\verb|C:\|(\verb|\|は\yen の意味) に移動する.\\
    \hspace*{10mm}\url{https://github.com/ketpic/kettex/releases}

\item そのまま解凍すると,\verb|KeTTeX-windows-20210322|フォルダができる.
\item \verb|kettexinst.cmd|をクリックする(詳細情報>実行)\\
・「ファイルが見つからない」のエラーは無視してよい.\\
・\verb|C:\|にkettexフォルダができる.\\
・\verb|KeTTeX-windows-20210322.zip|,\verb|KeTTeX-windows-20210322|を削除する.

\end{enumerate}

\item KeTCindyのインストール
\begin{itemize}
\item \verb|CTAN>ketcindy>repository|から\verb|Code>Download ZIP|で\verb|ketcindy-master.zip|をダウンロードする.
\verb|ketcindy-master|の\verb|-|を削除して,OneDriveの管理の外側におく.
\item \verb|doc>ketcindysettings.cdy|をクリック
\begin{enumerate}[(1)]
\item 左方にあるボタンで,言語,TeXの種類,描画コードを選ぶ.\\
\hspace*{10mm}ボタンを押すと順に項目が変わる.
\item 中央にあるボタンでTeXシステムを選ぶ.\\
\hspace*{10mm}KeTTeX,TeXLive以外の場合は,CindyScriptでパスを設定してから.\verb|Other|を選ぶ.
\item 右側にあるボタンを順に押す.\\
\hspace*{10mm}\verb|Mkinit|:初期化ファイルketcindy.iniをユーザホームに作成\\
\hspace*{10mm}\verb|Update|:TeXに入っているketlib関連のファイルを更新(コピー)\\
\hspace*{10mm}\verb|Work|:作業フォルダketcindy(+日付)をユーザホームに作成
\end{enumerate}
\item ketcindysettings.cdyのUpdateでエラーが出た場合\\
\hspace*{10mm}updaterフォルダにあるupdateketcindy.batを右クリックして管理者として実行する.
\end{itemize}

\item KeTCindyのテストラン
    \begin{enumerate}[(1)]
    \item ketcindysettings.cdyを終了してから,作業ディレクトリketcindy(+日付)を開く.
    \item \verb|templates|の1つのcdyファイルをダブルクリック.\\
      \hspace*{10mm}画面に白い枠が出れば,ライブラリの読み込みは成功.
    \item スクリーンの左上部にあるFigureボタンを押す.
    \item \TeX コンパイラが動いて,PDFが表示されれば成功.
   \item[注)] フォルダ名がハイフン(-)を含むと,コマンドプロンプトの窓が開かない場合がある.
  \end{enumerate}

\item TeXWorksの設定(kettexの場合)
  \begin{itemize}
  \item \url{http://www.tug.org/texworks/}からダウンロードする.
   \item TeXworksを立ち上げ,「TeXworks \verb|>| 環境設定 \verb|>| タイプセット」
  \item 上の欄(パス)に以下を追加\\
  \hspace*{5mm}\verb|C:\kettex\texlive\bin\win32|\\
  \hspace*{10mm}注) 上の行を上の欄の先頭になるように移動する.
  \item 下の欄の横にある + をクリック
    \begin{itemize}
    \item 名前:uplatex(ptex2pdf)またはplatex(ptex2pdf)
    \item プログラム : ptex2pdf
    \item 引数:\\
    \hspace*{10mm} \verb|-u|(uplatexの場合のみ)\\
    \hspace*{10mm} \verb|-l|\\
    \hspace*{10mm} \verb|-ot|\\
    \hspace*{10mm}  \verb|$synctexoption|\\
    \hspace*{10mm}  \verb|$fullname|
    \item[]OKボタンを押し,デフォルトを変更してOKボタンを押す.
    \end{itemize}
  \end{itemize}

\item MinGWのインストール
  \begin{itemize}
    \item 曲面描画のためには, gccが必要である.
    \item \url{https://sourceforge.net/projects/mingw-w64/}から,\verb|mingw-w64-install.exe|
    をダウンロードして実行
  \begin{enumerate}[(1)]
  \item Program FielesまたはProgram Files(x86)にインストールする.
  \item Architecture:Windows10が32bit版の場合\verb|i686|,64bit版の場合\verb|x86_64|を選択
  \end{enumerate}
  \end{itemize}

\item 手動でインストールする場合(KeTTeX)\vspace{2mm}\\
 \verb|ketcindysettings.cdy|のUpdateが実行できない場合は以下のようにする\\
 \hspace*{1zw}注)他のTeXの場合は,適宜パスを置き換える.\\
 \hspace*{4zw}\verb|C:\kettex\texlive| $=>$ \verb|C:\texlive\2020| など
\begin{enumerate}[(1)]
  \item \verb|ketcindy(-master)\ketcindyfolder|を開いておく.
  \item scriptsフォルダの中身を以下にコピーする.\\
 \verb|C:\kettex\texlive\texmf-dist\scripts\ketcindy|
  \item styleフォルダの中身を以下にコピーする.\\
 \verb|C:\kettex\texlive\texmf-dist\tex\latex\ketcindy|
  \item docフォルダの中身(figフォルダを除く)を以下にコピーする.\\
 \verb|C:\kettex\texlive\texmf-dist\doc\supports\ketcindy||
  \item コマンドプロンプトで以下を実行する\\
  \hspace*{1zw}\verb|C:\kettex\texlive\bin\win32\mktexlsr|
  \end{enumerate}
\end{enumerate}

\end{document}