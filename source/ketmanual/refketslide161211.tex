\documentclass[a4j]{jarticle}

\usepackage{/tools/ketpic,/tools/ketlayer}
\usepack

\input{/tools/reference/refstyle}

\setmargin{20}{20}{15}{25}

\begin{document}

\begin{flushright}
2016.12.11
\end{flushright}


\begin{center}
{\bf \huge \ketpic スタイル\ --ketslide--}\vspace{3mm}\\
\end{center}

\section{スタイル名}
\tab{ketslide.sty}

\section{概要}

プレゼン用スライドを作成する.\\
\hspace*{2zw}\bs documentclass[landscape,10pt]\{jarticle または article\}\\
\hspace*{2zw}スライドの大きさは 約130 $\times$ 85 (mm)


\section{マクロ一覧}

\subsection{プリアンブル用}

\tab{\bs ketmarginJ}jarticleのときのマージン設定

\tab{\bs ketmarginE}articleのときのマージン設定

\tab{\bs ketslideinit}パラメータ指定マクロへのdefault値設定

\noindent
\hspace*{2zw}\tab{\bs ketcletter}タイトルの文字色(black\\
\hspace*{2zw}\tab{\bs ketcbox}ボックス内の色(white\\
\hspace*{2zw}\tab{\bs ketdbox}ボックス内の色の濃さ(1)\\
\hspace*{2zw}\tab{\bs ketcframe}枠の色(black)\\
\hspace*{2zw}\tab{\bs ketcshadow}陰の色(black)\\
\hspace*{2zw}\tab{\bs ketdshadow}陰の色の濃さ(0.5)\\
\hspace*{2zw}\tab{\bs slidetitlex}タイトルの横位置(6)\\
%\hspace*{2zw}\tab{\bs slidetitley}タイトルの縦位置(7)\\
\hspace*{2zw}\tab{\bs slidetitlesize}タイトルの文字サイズ(2)\\
\hspace*{2zw}\tab{\bs mketcletter}メインタイトルの文字色(black)\\
\hspace*{2zw}\tab{\bs mketcbox}メインタイトルのボックス内の色(white)\\
\hspace*{2zw}\tab{\bs mketdbox}メインタイトルのボックス内の色の濃さ(1)\\
\hspace*{2zw}\tab{\bs mketcframe}メインタイトルの枠の色(white)\\
\hspace*{2zw}\tab{\bs mslidetitlex}メインタイトルの横位置(62)\\
%\hspace*{2zw}\tab{\bs mslidetitley}メインタイトルの立て位置(35)\\
\hspace*{2zw}\tab{\bs mslidetitlesize}メインタイトルの文字サイズ(3)\\
\hspace{6zw}注) これらのパラメータは本文中で \bs def を使って変更できる.


\subsection{本文用}

\tab[15zw]{\bs newslide[縦位置]\{タイトル\}}新しいスライドを始める.(縦位置のdefault=6)

\tab[15zw]{\bs sameslide}[縦位置]同じページとして,追加する.

\tab[15zw]{\bs mainslide[縦位置]\{タイトル\}}新しいメインスライドを始める.(縦位置のdefault=35)

\tab[15zw]{\bs mainsameslide[縦位置]}同じページとして,追加する.


\end{document}