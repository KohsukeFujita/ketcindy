\documentclass[a4j]{jarticle}

\usepackage{ketpic,ketlayer}

\usepack

\input{refstyle}

\setmargin{20}{20}{15}{25}

\begin{document}

\begin{flushright}
2016.12.11
\end{flushright}


\begin{center}
{\bf \huge \ketpic\ style ---ketlayer---}\vspace{3mm}\\
\end{center}

\section{スタイル名}
\tab{ketlayer.sty,\ ketlayer2e.sty(pict2e必要)}

\section{概要}
追記(コメント,メモ,挿絵)用の環境,マクロを定義.\\
\hspace*{5zw}ketpic.sty も読んでおく.\\
\hspace*{5zw}graphicx,\ color も必要

\section{環境}

\tab{\bs begin\br{layer}[方眼の水平移動]\br{W}\br{H}---\bs end\br{layer}}\\
\tab{}メモを書くためのpicture環境を定義して,方眼を描く.\\
\chuu 方眼の水平移動のデフォルトは 0\\
\chuu $\mathrm{H}=0$のときは,方眼を描かない.\\
\chuu $\mathrm{H}<0$のときは,上側に方眼を描く.\\
\rei \bs begin\br{layer}\br{170}\br{30}

\vspace{0.5\baselineskip}

\begin{layerv}{0}{30}

\end{layerv}

\vspace{7\baselineskip}

%\tab{\bs begin\br{layerv}[方眼の水平移動]\br{縦方向の移動}\br{H}---\bs end\br{layerv}}\\
%\tab{}縦方向の移動をし,水平方向は紙幅から自動計算.縦方向は下が+\\
%\chuu 該当部分の下におき,$-$で指定するのが普通

\tab{\bs begin\br{layerg}[1 or 0]\br{C}---\bs end\br{layerg}}

\tab{}C(図など)の下におき,Cの縦幅だけ方眼をおく\\
\chuu オプション引数(def=1):1のとき方眼を描き,0のとき描かない

\section{シンボル}


\begin{layer}{170}{0}

\putnotec{30}{35}{\cirscoremark{0.8}}

\putnotec{60}{35}{\scirscoremark{0.8}}

\putnotec{90}{35}{\triscoremark{0.8}}

\putnotec{120}{35}{\crosscoremark{0.8}}

\end{layer}


%\tab[6cm]{\bs circlemark[thickness]\br{size}}円\\
%\chuu サイズは直径10mmからの倍率で与える.\\
%\chuu thicknessのデフォルトは 8 ( milli inch)

%\tab[6cm]{\bs remarkmark[thickness]\br{size}}注意マーク

\tab[6cm]{\bs cirscoremark[thickness]\br{size}}手がきの2重丸

\tab[6cm]{\bs scirscoremark[thickness]\br{size}}手がきの単丸

\tab[6cm]{\bs triscoremark[thickness]\br{size}}手がきの三角

\tab[6cm]{\bs crosscoremark[thickness]\br{size}}手がきのバツ

\vspace{6\baselineskip}

\section{マクロ一覧}

\begin{layer}{160}{0}

\putnotee{120}{20}{\fbox{$\dfrac{1}{2}$}}

\end{layer}

\tab[14zw]{\bs putnotec\br{x}\br{y}\br{Char}}(x,\ y)を中心にCharを書く

\tab[14zw]{\bs putnotee\br{x}\br{y}\br{Char}}(x,\ y)の右にCharを書く

\tab[14zw]{\bs putnotew\br{x}\br{y}\br{Char}}(x,\ y)の左にCharを書く

\tab[14zw]{\bs putnotes\br{x}\br{y}\br{Char}}(x,\ y)の下にCharを書く

\tab[14zw]{\bs putnoten\br{x}\br{y}\br{Char}}(x,\ y)の上にCharを書く

\tab{\bs putnotene\br{x}\br{y}\br{Char}\ ,\ 
\bs putnotenw\br{x}\br{y}\br{Char}}

\tab{\bs putnotese\br{x}\br{y}\br{Char}\ ,\ 
\bs putnotesw\br{x}\br{y}\br{Char}}

\noindent
\rei putnotee\br{20}\br{5}\br{\bs fbox\br{\$\bs dfrac\br{1}\br{2}\$}}

\mbox{}

\begin{layer}{160}{0}

\lineseg[16]{135}{25}{30}{25}

\arrowlineseg[16]{130}{50}{10}{45}

\end{layer}

\tab{\bs lineseg[thickness]\br{$x$}\br{$y$}\br{$L$}\br{$\theta$}}\\
\tab{}点$(x, \ y)$から長さ$L$の線分を$\theta^{\circ}$
方向に描く(単位はmm)\\
\rei \bs lineseg[16]\br{130}\br{20}\br{30}\br{25}\\
\chuu thicknessの単位は milli inch(デフォルト=12)\\
\chuu $x,\ y,\ \theta$ は小数でもよい.

\tab{\bs dashlineseg[thickness]\br{$x$}\br{$y$}\br{$L$}\br{$\theta$}}\\
\tab{}点$(x, \ y)$から長さ$L$の破線を$\theta^{\circ}$
方向に描く(単位はmm)

\tab{\bs arrowlineseg[thickness]\br{$x$}\br{$y$}\br{$L$}\br{$\theta$}}\\
\tab{}矢印を描く(鏃は始点に描く)\\
\rei \bs arrowlineseg[16]\br{30}\br{20}\br{10}\br{45}

\tab[13zw]{\bs arrowhead[size]\br{$x$}\br{$y$}\br{$\theta$}}鏃だけを描く

\mbox{}

%\tab{\bs boxline[線幅]\br{x}\br{y}\br{W}\br{H}}\\
%\tab{}(x,\ y)の右上(ne)に幅W,高さHの矩形を描く.\\
%\chuu 線幅のデフォルトは8 (milli inch)\\
%\chuu n,\ s など8方向+c を付加すれば,それぞれの方向に描く\\
%\rei \bs boxlinen\br{30}\br{20}\br{50}\br{5}

\tab{\bs boxframe+dir[thickness]\br{x}\br{y}\br{W}\br{H}\br{文字}}\\
\tab{}(x,\ y)のdir方向に幅W,高さHの矩形を描き,中に文字を入れる\\
\chuu dir=n,\ s,\ e,\ w,\ ne,\ nw,\ se,\ sw,\ c \\
\rei \bs boxframen\br{30}\br{20}\br{50}\br{5}\br{}

%\tab{\bs dashboxline[線幅]\br{x}\br{y}\br{W}\br{H}}\\
%\tab{}(x,\ y)の右上(ne)に幅W,高さHの点線の矩形を描く.\\
%\chuu 線幅のデフォルトは8 (milli inch)\\
%\chuu 8方向+c については,\bs boxlineと同じ

\tab{\bs dashboxframe+dir[thickness]\br{x}\br{y}\br{W}\br{H}\br{文字}}\\
\tab{}(x,\ y)のdir方向にギザの矩形を描き,中に文字を入れる\\
\chuu dir=n,\ s,\ e,\ w,\ ne,\ nw,\ se,\ sw,\ c 

\tab{\bs jaggyboxframe+dir[thickness]\br{x}\br{y}\br{W}\br{H}\br{文字}}\\
\tab{}(x,\ y)のdir方向にギザ四角形を描き,中に文字を入れる\\
\chuu dir=n,\ s,\ e,\ w,\ ne,\ nw,\ se,\ sw,\ c 

\tab{\bs dialboxframe+dir[thickness]\br{x}\br{y}\br{W}\br{H}\br{文字}}\\
\tab{}(x,\ y)のdir方向にダイヤ型からなる矩形を描き,中に文字を入れる\\
\chuu dir=n,\ s,\ e,\ w,\ ne,\ nw,\ se,\ sw,\ c 

\tab{\bs eraser+dir[枠]\br{$x$}\br{$y$}\br{W}\br{H}}\\
\tab{}$(x,\ y)$のdir方向の長方形の内部を消す\\
\chuu 枠=0 とすると枠線を描かない(デフォルトは 1)\\
\chuu dir=n,\ s,\ e,\ w,\ ne,\ nw,\ se,\ sw,\ c 

\tab{\bs shadebox+dir[枠描画]\br{$x$}\br{$y$}\br{W}\br{H}\br{D}\br{色}}\\
\tab{}$(x,\ y)$のdir方向に幅W,高さHの矩形の内部を濃さDで塗る\\
\chuu 枠描画のデフォルトは 0(枠線を描かない)\\
\chuu dir=n,\ s,\ e,\ w,\ ne,\ nw,\ se,\ sw,\ c 

\vspace{\baselineskip}

\begin{layer}{160}{0}

\boxframese[24]{10}{0}{30}{16}{boxframe}
\dashboxframese{50}{0}{30}{16}{dashboxframe}
\jaggyboxframese{90}{0}{30}{16}{jaggyboxframe}
\diaboxframese{130}{0}{30}{16}{diaboxframe}
\shadeboxse{10}{0}{30}{16}{0.5}{green}

\end{layer}

\newpage

\tab{\bs popframe[thickness]\br{x}\br{y}\br{Ds}\br{色s}\br{Dp}\br{色p}\br{色f}\br{文字}}\\
\tab{}(x,\ y)の右下(se)に文字入り矩形を描き,濃さDsの陰をつける\\
\chuu 色p : 背景色,色f : 枠の色\\
\chuu 矩形の大きさは文字から自動計算する\\
\chuu 線の太さ(thickness)のデフォルトは8\\
\chuu 文字列の幅$\leqq 200mm$,高さ$\leqq 100mm$\\
\rei \bs popframe\br{0}\br{5}\br{0.5}\br{black}\br{1}\br{yellow}\br{yellow}\br{タイトル}

\tab{\bs colorframe[thickness]\br{x}\br{y}\br{Dp}\br{色p}\br{色f}\br{文字}}\\
\tab{}(x,\ y)の右下(se)に文字入り矩形を描き,背景を色pで塗る\\
\chuu 矩形の大きさは文字から自動計算する\\
\chuu 線の太さ(thickness)のデフォルトは8\\
\chuu 文字列の幅$\leqq 200$mm,高さ$\leqq 100$mm\\
\rei \bs colorframe\br{100}\br{5}\br{0.3}\br{green}\br{blue}\br{文字列}

\begin{layer}{160}{0}
\popframe{60}{5}{0.3}{black}{1}{yellow}{yellow}{\Large タイトル}
\colorframe{100}{5}{0.3}{green}{blue}{\Large\bf 強調文字}
\end{layer}

\vspace{15mm}

\begin{layer}{160}{0}

\hjaggyline{130}{10}{25}

\vjaggyline{140}{20}{15}

\end{layer}

%\vspace{6\baselineskip}

\tab{\bs hjaggyline[thickness]\br{x}\br{y}\br{W}}\\
\tab{}(x,\ y)から左に幅Wのギザ線を描く.\\
\chuu bを付加すると,線の出方が逆になる.

\tab{\bs vjaggyline[thickness]\br{x}\br{y}\br{W}}\\
\tab{}(x,\ y)から下に幅Wのギザ線を描く.\\
\chuu bを付加すると,線の出方が逆になる.

\tab{\bs circleline[thickness]\br{x}\br{y}\br{size}}\\
\tab{}(x,\ y)を中心に
円を描く

\tab{\bs ballonr[thickness]\br{x}\br{y}\br{size}\br{Char}}\\
\tab{}(x,\ y)から右上に吹き出しとCharを描く

\tab{\bs ballonl[thickness]\br{x}\br{y}\br{size}\br{Char}}\\
\tab{}(x,\ y)から左上に吹き出しとCharを描く


\tab[14zw]{\bs lefthand[thickness]\br{x}\br{y}}(x,\ y)に指先を描く

\tab[14zw]{\bs righthand[thickness]\br{x}\br{y}}(x,\ y)に指先を描く

\tab[16zw]{\bs leftdownhand[thickness]\br{x}\br{y}}(x,\ y)に指先を描く

\tab[16zw]{\bs rightdownhand[thickness]\br{x}\br{y}}(x,\ y)に指先を描く

\begin{layer}{180}{0}

\ballonr{30}{40}{1}{Example1}

\ballonl{90}{40}{1}{Example2}

\lefthand{120}{30}

\righthand{140}{30}

\leftdownhand{120}{25}

\rightdownhand{140}{25}

\end{layer}



\end{document}
